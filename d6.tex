\documentclass[12pt]{article}
\usepackage{geometry}
\usepackage{fullpage}
\usepackage{graphicx}
\geometry{a4paper}
\setlength\parindent{0pt}
\title{Deliverable 6}
\author{
    Gordon Reid: 1002536R\\
    Ryan Wells: 1002253W\\
    Kristopher Stewart: 1007175S\\
    David Selkirk: 1003646S\\
    James Gallagher: 0800899G\\
}
\date{\today}

\begin{document}

\maketitle

\newpage

\tableofcontents

\newpage

\section{Introduction}

\newpage

\section{System Architecture}
\subsection{Diagram}
\begin{centering}
  \includegraphics[width=0.7\textwidth,angle=270]{PSDDiagram.pdf}
\end{centering}
\subsection{Rationale}
We decided upon this system architecture as any component of the system should have the properties of being interchangeable with another substitutable component (substitutable defined as being having the same API). For this reason, we grouped relatable functionalities together and made them into a component. The ``Ad Manager'' and the ``Databases'' components could be combined together into one large ``Database'' component, but we did not want to create a coupling between these components - the ``Ad Manager'' may well have a database inside it, but the information inside this database is managed in a different way from the other data in the system such as users and companies.
All other components are defined by the primary reason that any component in the system can be replaced with no effect to the overall working of the system.
\newpage

\section{State Charts}

\subsection{Advertisement}

\subsubsection{Diagram}

\includegraphics{advertState.png}

\subsubsection{Rationale}

\subsection{Student}

\subsubsection{Diagram}

\includegraphics[width=\textwidth]{studentState.png}

\subsubsection{Rationale}

\newpage

\section{Class Diagrams and API Specifications}

\subsection{Advert Manager}

\subsubsection{Class Diagram}

\subsubsection{API Specification}

\subsection{Session}

\subsubsection{Class Diagram}

\subsubsection{API Specification}

None. Possibly extend this to include Main/run?

\subsection{User Interface}

\subsubsection{Class Diagram}

\subsubsection{API Specification}

\subsection{Admin}

\subsubsection{Class Diagram}

\subsubsection{API Specification}

\textbf{Full name:} public abstract interface Admin\\

\textbf{Package:} uk.ac.glasgow.internman.admin

This is the interface for the administration part of the database, accessable by
only the course coordinator.
This assumes that the user using this interface has Course Coordinator
rights. \\ 
\begin{itemize}

\item{\textbf{public void approveAdvertisement(Advertisement advertisement)}

Approves an advertisement.

\textbf{Preconditions:} Advertisement must be in advertisement database.

\textbf{Invariants:}

\textbf{Postconditions:} Advertisement is now approved.}

\item{\textbf{public boolean addCompany(String name, String email, String
    password)} 

Interface to add a company to the database, fuctionality is delegated to the
database component.
Error checking is done inside this function, but not whether or not a company
currently exists inside the companies database.

\textbf{Preconditions:} 

\textbf{Invariants:}

\textbf{Postconditions:} The return value of the delegated function indicates
the success of this function.}

\item{\textbf{public void approveAcceptedOffer(String matriculation)}

Approves the offer most recently accepted by the student with this matriculation
id. 
\textbf{Preconditions:} Student with this matricualtion must have a successful
application and must have notified the Course Coordinator of their success.

\textbf{Invariants:} Student must not accept another successful application
during this review process.

\textbf{Postconditions:} Students status is changed to a success status.}

\item{\textbf{public void assignAcademinVisitor(String matriculation, String
    visitorName)}
Sends email to student, visitor and employer manager to let them know about
assignment. 

\textbf{Preconditions:}

\textbf{Invariants:} 

\textbf{Postconditions:} }

\end{itemize}
\subsection{Login Prompt}

\subsubsection{Class Diagram}

\subsubsection{API Specification}

\subsection{Database}

\subsubsection{Class Diagram}

\subsubsection{API Specification}

\textbf{Full name:} public abstract interface UserDatabase\\

\textbf{Package:} uk.ac.glasgow.internman.users

This is the interface for the user database. This database will hold information
on students, companies, and the course coordinator. For University students and
staff members, the database will access the MyCampus GUID system for login. For
companies, a separate store for their user information will be used.\\

\textbf{Methods}

\begin{itemize}

\item{\textbf{public boolean addCompany(String name, String email, String 
password)}

Add information about a company to the database.

\textbf{Preconditions:} Database must not already contain details of the 
company.

\textbf{Invariants:}

\textbf{Postconditions:} Database now contains information about a company 
including their login password.}

\item{\textbf{public Company getCompany(String name)}

Gets information about the company with the given name.

\textbf{Preconditions:}

\textbf{Invariants:}

\textbf{Postconditions:} If a valid name is given, the object associated with
the company is returned, otherwise null is returned.}

\item{\textbf{public Student getStudent(String guid)}

Gets information about the student with the given GUID from MyCampus and the
application's user specific information database (such as internship progress).

\textbf{Preconditions:}

\textbf{Invariants:}

\textbf{Postconditions:} If a valid GUID is given, the object associated with
the student is returned, otherwise null is returned. If this is the first time
a student has been asked for, a record will be added to the user specific
information database.}

\item{\textbf{public void updateStudent(Student student)}

Updates the user specific data for the supplied student. For example, when a
student obtains an internship, or has had their internship assessed.

\textbf{Preconditions:} Student must be a valid Computing Science student and
is known to the application's user database.

\textbf{Invariants:}

\textbf{Postconditions:} Student's user specific information is up to date.}

\end{itemize}

\newpage

\section{Acceptance Test Plan}

\end{document}
